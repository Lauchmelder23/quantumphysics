% !TeX root = ../script.tex
\documentclass[../../script.tex]{subfiles}

\begin{document}
    \chapter{Fundamentals of Quantum Physics}
    \minitoc
    \vspace*{\fill}\par
    \pagebreak  

    Due to the uncertainty principle, location and impulse of an atomic particle cannot be both stated with arbitrairy precision.
    The classic trajectory, represented in the model of mass points by a well-defined curve in space $\vec{r}(t)$, is replaced by the probability
    \begin{equation}
        W(x, y, z, t) \dd{v} = \abs{\psi(x, y, z, t)}^2 \dd{V}
    \end{equation}
    to find the particle in the volume element $\dd{V} = \dd{x}\dd{y}\dd{z}$ at the time $t$. This probability depends on the absolute square of the matter wave function $\psi(x, y, z, t)$.

    In this chapter we wnat to show how this wave function can be calculated for simple examples. These examples will also demonstrate the physical fundamentals of quantum mechanics and its
    differences to classical particle mechanics, elaborate on the concept of \textit{quantum numbers} and show under which conditions quantum mechanical results can be transitioned into classical physics.
    This is supposed to clarify that classic (i.e.\ pre-quantum) mechanics are contained in quantum mechanics as a limiting case for very small de Broglie wavelengths $\lambda_{dB} \rightarrow 0$.

    These examples should also demonstrate that almost all insights of quantum mechanics are already known in classic wave optics. This means: the actual novel concepts in quantum mechanics
    is the description of classic particles with matter waves. The deterministic description of the temporal development of location and impulse of a particle is thus replaced with a statistical
    treatment, by which we can only discuss probabilities of the results of a measurement. A fundamental uncertainty occurs when we observe location and impulse at the same time.

    \subfile{schroedinger.tex}
    \subfile{examples.tex}
\end{document}