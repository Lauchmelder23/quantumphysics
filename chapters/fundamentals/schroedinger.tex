% !TeX root = ../../script.tex
\documentclass[../../script.tex]{subfiles}

\begin{document}
\section[The Schrödinger Equation]{The Schrödinger Equation}

In this section we will outline the fundamental equation of quantum mechanics, which was established by \textit{Erwin Schrödinger} (1887--1961) in 1926.
The solutions of this equation are the desired wave functions $\psi(x, y, z, t)$. However, these solutions can only be derived in analytical form for a few very simple physical problems.
Very fast computers are usually able to numerically compute solutions for complex problems.

First, we will consider the mathematically simplest case.A free particle of mass $m$ which moves at a constant velocity $\vec{v}$ in direction $x$.
With $\vec{p} = \hbar\vec{k}$ and $E = \hbar \omega = \ekin$ (because $\epot = 0$), we know the wave function must be of the form 
\begin{equation}\label{eq:wavefuncform}
	\psi(x, t) = Ae^{i(kx-\omega t)} = Ae^{(i/\hbar)(px-\ekin t)}
\end{equation}
Here we use the fact that $\ekin = p^2 / 2m$, which is the kinetic energy of the particle. Since the mathematical representation is absolutely identical to that of an electromagnetic wave,
it makes sense to start with the wave equation 
\begin{equation}\label{eq:1dwaveeq}
	\pdv[2]{\psi}{x} = \frac{1}{u^2} \pdv[2]{\psi}{t}
\end{equation}
for waves propagating with phase velocity $u$ in direction $x$. For \textit{stationary} problems where $\vec{p}$ and $E$ are time-independent the wave function can be split into a 
strictly location-dependent factor $\psi(x) = Ae^{ikx}$, and a strictly time-dependent phase factor $e^{-i\omega t}$. Thus we can write 
\begin{equation}\label{eq:splitwave}
	\psi(x, t) = \psi(x) \cdot e^{-i\omega t} = A e^{ikx} \cdot e^{-i\omega t}
\end{equation}
If we use the ansatz~\eqref{eq:splitwave} in the wave equation~\eqref{eq:1dwaveeq}, and the fact that $k^2 = p^2 / \hbar^2 = 2m\ekin/\hbar^2$, we get the equations
\begin{equation}\label{eq:values}
	\begin{split}
		&\pdv[2]{\psi}{x} = -k^2\psi = -\frac{2m}{k} \cdot \ekin \cdot \psi \\
		&\pdv[2]{\psi}{t} = -\omega^2 \psi
	\end{split}
\end{equation}
Comparison with~\eqref{eq:1dwaveeq} gives us 
\[
	u^2 = \frac{\omega^2}{k^2} \implies u = \frac{\omega}{k}
\]
Note that the particle velocity $v_\text{P} = v$ 
\[
	v = \frac{p}{m} = \frac{\hbar k}{m} = \pdv{\omega}{k}
\]
is different from the phase velocity $u = v_{\text{ph}} = \omega / k$.

In the general case the particle can move in a force field. If it is conservative then we can assign each point a potential energy, with the condition that the total energy $E = \ekin + \epot$ remains constant.
Using $\ekin = E - \epot$ and~\eqref{eq:values} we then receive the one-dimensional stationary Schrödinger equation
\begin{tcolorbox}[ams equation]
	\frac{-\hbar^2}{2m} \pdv[2]{\psi}{x} + \epot\psi = E\psi
\end{tcolorbox}
For the general case where the particle is moving freely in three-dimensional space we can use the three-dimensional wave equation 
\[
	\laplacian\psi = \frac{1}{u^2} \pdv[2]{\psi}{t}	
\]
and the ansatz $\psi(x, y, z, t) = \psi(x, y, z) \cdot e^{-i\omega t}$, we can establish the three-dimensional stationary Schrödinger equation 
\begin{tcolorbox}[ams equation]\label{eq:stationaryschroedinger}
	\frac{-\hbar^2}{2m} \laplacian \psi = \epot \psi = E \psi
\end{tcolorbox}
If we differentiate~\eqref{eq:wavefuncform} partially for time we receive 
\[
	\pdv{\psi}{t} = -\frac{i}{h} \ekin \cdot \psi	
\]
and with~\eqref{eq:values} we can find the time-dependent equation for a free particle with $\epot = 0$ (i.e. $\ekin = \const$)
\begin{equation}
	-\frac{\hbar^2}{2m} \pdv[2]{\psi(x, t)}{x} = i\hbar \pdv{\psi(x, t)}{t}
\end{equation}
The three-dimensional representation is then 
\begin{tcolorbox}[ams equation]
	-\frac{\hbar^2}{2m} \laplacian \psi(\vec{r}, t) = i\hbar \pdv{\psi(\vec{r}, t)}{t}
\end{tcolorbox}
There are some remarks to be made:
\begin{itemize}
	\item In this ``derivation'' we have used the de Broglie-relationship $\vec{p} = \hbar \vec{k}$, which is only supported by experiments and has no mathematical justification.
	\item The law of conservation of energy of quantum mechanics is $E\psi = \ekin\psi + \epot\psi$. Like in classical mechanics, there is no derivation for this law, and is accepted as truth from experience.
	\item While electromagnetic waves have a linear dispersion relation $\omega(k) = kc$, the matter wave $\psi(\vec{r}, t)$ of a free particle has a \textit{quadratic} dispersion relation $\omega(k) = (\hbar/2m) \cdot k^2$. This results from $E = \hbar\omega = p^2/2m$.
	\item The Schrödinger equations are a \textit{linear} homogeneous differential equation. Because of this, different solutions of the equation can be superpositioned. This means, if $\psi_1$ and $\psi_2$ are solutions to the equation, then $\psi_3 = a\cdot\psi_1 + b\cdot\psi_2$ is also a solution.
	\item Since the time-dependent Schrödinger equation is a complex equation, the wave functions $\psi$ may also be complex. The absolute square $\abs{\psi}^2$ however, which represents the probability of the presence of a particle, is always real.
\end{itemize}
For non-stationary problems (i.e. $E = E(t)$ and $p = p(t)$), the dispersion relation $\omega(t)$ also becomes time-dependent. This means that $\partial^2\psi/\partial t^2$ can no longer be written as $-\omega^2 \psi$, and cannot be derived from the wave equation for matter waves of particles.

Schrödinger postulated (!), that even for time-dependent potential energy $\epot(\vec{r}, t)$ the equation 
\begin{tcolorbox}[ams equation]\label{eq:schroedinger}
	\frac{-\hbar^2}{2m} \laplacian \psi(\vec{r}, t) + \epot(\vec{r}, t) \psi(\vec{r}, t) = i\hbar\pdv{\psi(\vec{r}, t)}{t}
\end{tcolorbox}
holds. The general time-dependent Schrödinger equation has since been verified in numerous experiments, and is generally considered correct, even if no mathematical justification exists.
This equation is the fundamental equation of quantum mechanics.

For stationary problems we can separate $\psi(\vec{r}, t)$ into $\psi(\vec{r}, t) = \psi(\vec{r}) \cdot e^{-i(E/\hbar) \cdot t}$. Inserting this into~\eqref{eq:schroedinger} yields the 
stationary Schrödinger equation~\eqref{eq:stationaryschroedinger} for $\psi(\vec{r})$.
\end{document}