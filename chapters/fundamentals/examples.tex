% !TeX root = ../../script.tex
\documentclass[../../script.tex]{subfiles}

\begin{document}
\section[Examples of the Stationary Schrödinger Equation]{Examples of the Stationary Schrödinger Equation}

We now want to solve the Schrödinger equation
\[
	\frac{-\hbar^2}{2m} \dv[2]{\psi}{x} + \epot\psi = E\psi
\]
for a few simple, one-dimensional problems. These examples will illustrate the description of classical particles as waves and the following physical consequences.

\subsection{The Free Particle}

A particle is said to be free, if it is moving in a constant potention $\phi_0$, because then $\vec{F} = -\grad{\epot}$ means that no forces are acting on the particle. 
Through a suitable choice of the zero point energy we can set $\phi_0 = 0$, i.e. $\epot = 0$, and thus get the Schrödinger equation for a free particle 
\begin{equation}\label{eq:onedschroedinger}
	\frac{-\hbar^2}{2m} \dv[2]{\psi}{x} = E\psi
\end{equation}
The total energy $E = \ekin + \epot$ is because of $\epot$ now 
\[
	E = \frac{p^2}{2m} = \frac{\hbar^2 k^2}{2m}
\]
Thus~\eqref{eq:onedschroedinger} gets reduced to 
\[
	\dv[2]{\psi}{x} = -k^2 \psi
\]
which has the general solution 
\begin{equation}\label{eq:onedsolution}
	\psi(x) = Ae^{ikx} + Be^{-ikx}
\end{equation}
The time-dependent wave function 
\begin{equation}\label{eq:onedsolutioncomplete}
	\psi(x, t) = \psi(x) \cdot e^{-i\omega t} = Ae^{i(kx - \omega t)} + Be^{-i(kx + \omega t)}
\end{equation}
represents the superposition of a planar wave travelling in the $+x$ and $-x$ direction.

The coefficients $A$ and $B$ are the amplitudes of those waves, which are determined by the boundary conditions.
For example, the wave function of electrons which are emitted from a cathode in $+x$ direction towards a detector, will have $B = 0$, since there are no particles moving in $-x$.
From this experimental setup we know that the electrons are found along the length $L$ of the path between cathode and detector. This means their wave function can only be different from zero in this region of space.
Using the normalization condition we get 
\begin{align*}
	&\int_0^L \abs{\psi(x)}^2 \dd{x} = 1 \\
	&\implies A^2 \cdot L = 1 \implies A = \frac{1}{\sqrt{L}}
\end{align*}
To determine the location of a particle at time $t$ more accurately, we will have to construct \textit{wave packets} in place of planar waves~\eqref{eq:onedsolution}
\begin{equation}
	\psi(x, t) = \int_{k_0 - \Delta k / 2}^{k_0 + \Delta k / 2} A(k) e^{i(kx - \omega t)} \dd{k}
\end{equation}
The location uncertainty of this packet at $t = 0$ is 
\[
	\Delta x \ge \frac{\hbar}{2 \Delta p_x} = \frac{1}{2\Delta k}
\]
and depends on the pulse width $\Delta p_x = \hbar \Delta k$. The larger $k$ is, the more certainly $\Delta x(t = 0)$ can be determined, but the faster the wave packet spreads.

Experimentally, this can be illustrated as follows: If we apply a short voltage pulse to the cathode at time $t = 0$, then electrons can start travelling towards the detector at this instance.
The emitted electrons have a velocity distribution $\Delta v$, such that electrons with differing velocities $v$ will not necessarily be in the same location $x$ at a later point in time $t$.
Instead they are spread over the interval $\Delta x(t) = t \cdot \Delta v$. The velocity distribution is described by $\Delta v \propto \Delta k$ of the wave packet, such that the location uncertainty $\Delta x$ 
\[
	\dv{(\Delta x(t))}{t} = \Delta v(t = 0) = \frac{\hbar}{m} \Delta k(t = 0)
\]
changes proportionally to the initial impulse uncertainty.

\subsection{Potential Step}

We are still considering the particles from the previous example, however we introduce a potential step at $x = 0$. This means we are considering the potential 
\[
	\phi(x) = \begin{cases}
		0 & x < 0 \\
		\phi_0 & x \ge 0
	\end{cases}
\]
This means the particles are still moving in direction $+x$ and are free ($\epot = 0$) if their position is $x < 0$. However at position $x = 0$ they enter into an area of higher potential $\phi(x \ge 0) = \phi_0 > 0$.
The potential energy in this area is still constant $\epot = \energy{0}$. Thus, at $x = 0$ we have a potential jump $\Delta E = \energy{0}$.
This problem has an equivalent in classic optics: a planar lightwave encountering a boundary between vacuum and material (e.g.\ a glass surface).

We divide the domain $-\infty < x < +\infty$ into two areas I and II\@. For area I with $\epot = 0$ we still have the equation~\eqref{eq:onedschroedinger} with the solution~\eqref{eq:onedsolution}
for the location part of the wave function 
\[
	\psi_{\text{I}}(x) = Ae^{ikx} + Be^{-ikx}
\]
where $A$ is the amplitude of the incidental wave, and $B$ the amplitude of the wave reflected from the potential step.

\textbf{Note:} The complete solution is~\eqref{eq:onedsolutioncomplete}. The temporal part of the soltuion is often omitted, because it has no influence in the stationary problems considered here.

In area II, the Schrödinger equation becomes
\begin{equation}
	\dv[2]{\psi}{x} + \frac{2m}{\hbar^2}(E - \energy{0})\psi = 0
\end{equation}
If we use the shorthand $\alpha = \sqrt{2m(\energy{0} - E)} / \hbar$ we can reduce the equation to 
\begin{equation}\label{eq:potentialstep}
	\dv[2]{\psi}{x} - \alpha^2\psi = 0
\end{equation}
This equation has the solution 
\begin{equation}\label{eq:potentialstepsolution}
	\psi_{\text{II}} = Ce^{+\alpha x} + De^{-\alpha x}
\end{equation}
If 
\[
	\psi(x) = \begin{cases}
		\psi_{\text{I}} & x < 0 \\
		\psi_{\text{II}} & x \ge 0
	\end{cases}
\]
is a solution to the Schrödinger equation~\eqref{eq:potentialstep} on the entire domain $-\infty < x < +\infty$, then $\psi$ has to be continuously differentiable at every point,
or else the second derivative $\dd^2\psi / \dd x^2$ is not defined, and thus the Schrödinger equation is not applicable.
Using~\eqref{eq:onedsolution} and~\eqref{eq:potentialstepsolution} this results in the boundary conditions
\begin{subequations}\label{eq:boundaryconditions}
	\begin{equation}
		\begin{split}
			\psi_{\text{I}}(x = 0) &= \psi_{\text{II}}(x = 0) \\
			&\implies A + B = C + D
		\end{split}
	\end{equation}

	\begin{equation}
		\begin{split}
			\eval{\dv{\psi_{\text{I}}}{x}}_0 &= \eval{\dv{\psi_{\text{II}}}{x}}_0 \\
			&\implies ik(A - B) = \alpha (C - D)
		\end{split}
	\end{equation}
\end{subequations}
We can now investigate the two cases where the energy $\ekin = E$ of the incoming particle is smaller or larger than the potential step.

\subsubsection{(a) $E < \energy{0}$}
In this case, $\alpha$ is real valued and the coefficient $C$ in~\eqref{eq:potentialstepsolution} must be zero, because otherwise
\[
	\psi_{\text{II}} \xrightarrow{x \rightarrow +\infty} \pm\infty
\]
If this happens, the wave function is not normalizable. With the above boundary conditions this yields
\begin{align}
	B = \frac{ik + \alpha}{ik - \alpha}A && \text{and} && D = \frac{2ik}{ik - \alpha} A
\end{align}
Thus the wave function for $x < 0$ becomes 
\begin{equation}
	\psi_{\text{I}}(x) = A \left[ e^{ikx} + \frac{ik + \alpha}{ik - \alpha}e^{-ikx} \right]
\end{equation}
The fraction of reflected particles is calculated as 
\begin{equation}
	R = \frac{\abs{Be^{-ikx}}^2}{\abs{Ae^{ikx}}^2} = \frac{\abs{B}^2}{\abs{A}^2} = \abs{\frac{ik + \alpha}{ik - \alpha}}^2 = 1
\end{equation}
which means that \textit{all} particles are being reflected if $E < \energy{0}$. This corresponds to the expected classical behaviour or particles.
However there is a notable difference to classic particle mechanics:
\begin{tcolorbox}
	The particles are not being reflected at $x = 0$, but instead penetrate the domain $x > 0$ where $\epot = \energy{0} > \ekin$ before returning,
	even if their energy $\ekin < \energy{0}$ should not be enough to do so in a classical model.
\end{tcolorbox}
The probability $P(x)$ of finding a particle in $x > 0$ is 
\begin{equation}
	P(x) = \abs{\psi_{\text{II}}}^2 = \abs{De^{-\alpha x}}^2 = \frac{4k^2}{\alpha^2 + k^2} \abs{A}^2 e^{-2\alpha x} = \frac{4k^2}{k_0^2} \abs{A}^2 e^{-2\alpha x}
\end{equation}
where $k_0^2 = 2m\energy{0} / \hbar^2$. After a distance $x = 1/(2\alpha)$, the penetration probability is reduced to $1/e$ of its value at $x = 0$.

We already know this result from wave optics. Even if waves are reflected totally at a boundary with refraction index $n = n' - i\kappa$, the wave penetrates the surface of the medium before returning, and the intensity
of the penetrating wave sinks to $1/e$ of its initial value after a distance $x = 1/(2k\kappa) = \lambda/(4\pi\kappa)$.
\begin{tcolorbox}
	Particles with energy $E$ can penetrate into potential areas $\energy{0} > E$ with a certain probability, even if they shouldn't be able to according to classic particle physics.
\end{tcolorbox}
Once we accept that particles are described by waves, we can come to the conclusion that particles are allowed to exist in \textit{classically forbidden} locations.
\end{document}